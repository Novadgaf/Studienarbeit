%!TEX root = ../dokumentation.tex

\pagestyle{empty}

\renewcommand{\abstractname}{Zusammenfassung}
\begin{abstract}
Schach erlebt in den letzten Jahren einen großen Aufschwung. Aufgrund von Corona, Streaming-Plattformen und Webseiten, 
auf denen online gespielt werden kann, steigt die Zahl an Spielern täglich. In dieser Arbeit wird der Prozess der Erstellung einer 
Schachapplikation aufgefasst, die mithilfe von einer Kamera Spielzüge auf einem echten Spielbrett erkennt und diese dann an den Computer übermittelt 
und es ebenfalls ermöglicht gegen eine Engine zu spielen. Die Applikation wurde mit Python unter Verwendung von Pygame und OpenCV erstellt. 
Im Zuge dessen werden Grundlagen der digitalen Bildverarbeitung und Algorithmen von Schachcomputern, sowie deren Geschichte und Entwicklung aufgezeigt. 
Ebenfalls werden Ansätze, Probleme und Lösungswege diskutiert, die zur Entwicklung der fertigen Applikation geführt haben.
\end{abstract}

\renewcommand{\abstractname}{Summary}
\begin{abstract}
Chess has experienced a great upswing in recent years. Due to Corona, streaming platforms and websites, 
where online games can be played, the number of players is increasing daily. In this thesis, the process of creating a 
chess application that uses a camera to detect moves on a real game board and then transmits them to the computer, and also allows 
to play against an engine. The application was created with Python using Pygame and OpenCV. 
In the course of this, basics of digital image processing and algorithms of chess computers, as well as their history and development are shown. 
Also discussed are approaches, problems and solutions that led to the development of the final application.
\end{abstract}
