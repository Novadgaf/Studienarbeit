\chapter{Fazit und Ausblick}
\section{Fazit}
Die Applikation ist im Ganzen ein Erfolg. Die Ziele wurden erreicht, jedoch gab es auch einige Hürden. 
Vor allem im anfänglichen Entwicklungsprozess wurden viele Fehler gemacht und Zeit verschwendet, was verhindert werden konnte. 
Grund dafür war das zu schnelle Starten in die Programmierung ohne ausreichende Vorüberlegungen. Dies führte zu einem Verwerfen vieler 
Arbeit, die dann mehrmals gemacht werden musste. Außerdem, wurde an einigen Stellen, wie der Erstellung einer eigenen Engine viel Zeit verschwendet bei dem 
Versuch durch Optimierung der Suchalgorithmen eine schnellere Suche zu entwickeln, anstatt die Ineffizienz der Zugvalidierung, die der eigentliche 
Auslöser war, anzugehen.  

\section{Ausblick}
\subsection{Online Funktionalität}
Weitere Verbesserungen der Applikation, ist die Verbindung mit Online Portalen, um mit anderen Spielern auf der ganzen Welt spielen zu können.
Im jetzigen Stadium ist es nur möglich gegen eine Engine zu spielen. Mithilfe weiteren Bibliotheken zur Interaktion mit dem Internet oder sogar einem 
eigenen Server kann dies erreicht werden.

\subsection{Speichern der Spiele}
Im jetzigen Stadium der Applikation ist nur das Spielen selber implementiert. Das Abspeichern und anschließende abrufen der Spielzüge erlaubt es dem Nutzer 
sich Fehler wieder anzuschauen und so eine bessere Lernkurve zu erzielen.

\subsection{Spiel auf Zeit}
Schach wird meistens Mithilfe einer Schachuhr gespielt. Diese schränkt die Bedenkzeit beider Spieler ein und bietet eine weitere Herausforderung. Der Grund 
für die noch fehlende Implementierung ist der Zusammenhang mit der Kamera. Wenn der Spieler immer erst den Zug des Gegners spielen muss und dann 
anschließend seinen Zug macht und diesen dann erst über die Kamera aufnimmt, verliert er zu viel Zeit, als dass es Sinn ergibt eine Uhr zu 
implementieren. Durch eine Verbesserung der Zugerkennung, die eventuell automatisch statt manuell läuft, kann dieses Problem umgangen werden.

\subsection{Weitere Spielmodi}
Neben dem klassischen Schach gibt es einige weitere Spaß Modi, die von der Schach-Community gerne gespielt werden. Zurzeit sind diese nicht implementiert aber
vor allem mit Blick auf eine Online Implementation ergibt es Sinn, das Programm um diese zu erweitern.