%!TEX root = ../dokumentation.tex

\chapter{Diskussion über die Benutzeroberfläche und Programmiersprache}
Dieses Kapitel beschäftigt sich  mit den relevanten Entscheidungen, die hinter dem Aufbau und der Implementierung der Anwendung stehen. 
Die zentralen Auswahlkriterien, wie die verwendete Programmiersprache und die verwendeten Softwarebibliotheken, 
bestimmen in hohem Maße die Eigenschaften und Fähigkeiten des Programms.

Zunächst wird die Begründung für die Auswahl von Python als bevorzugter Sprache analysiert, 
wobei Alternativen wie Java und C++ in Betracht gezogen werden. Danach rückt die Untersuchung der während der Entwicklungsphase 
des Projekts genutzten Bibliotheken in den Fokus. Pygame wird für die Erstellung der Spielumgebung verwendet und OpenCV für die Bildverarbeitung. 
Die Vor- und Nachteile jeder Bibliothek werden gegen potenzielle Alternativen abgewogen und geben umfassende Einblicke in den Entscheidungsprozess.

Außerdem beschäftigt sich das Kapitel mit dem \ac{UI} Designs und der \ac{UX}, 
Prinzipien, die die Benutzerbindung und -zufriedenheit steigern. Diese Komponenten sind für den Erfolg jeder Anwendung unerlässlich, 
insbesondere in interaktiven Umgebungen wie einem Spiel.

Der abschließende Abschnitt fasst die Diskussionen zusammen und bekräftigt die Begründungen für die Auswahl, indem er reflektiert, 
wie diese Entscheidungen es ermöglicht haben, die funktionalen Anforderungen der Anwendung zu erfüllen und gleichzeitig ein 
zufriedenstellendes Benutzererlebnis zu gewährleisten.

\section{Auswahl der Programmiersprache}
Eine der grundlegendsten Entscheidungen bei der Entwicklung von Software ist die Wahl der Programmiersprache. 
Die Entscheidung für Python wurde durch eine Reihe von Faktoren beeinflusst, welche im Folgenden aufgezeigt werden. 
Jede Programmiersprache hat Stärken und Schwächen und die Wahl hängt von den spezifischen Anforderungen des Projekts ab.

\subsection{Python}
Python, die für dieses Projekt gewählte Sprache, ist eine High-Level-Interpretersprache, die für ihre Einfachheit und Benutzerfreundlichkeit bekannt ist.
Sie wird zunehmend für eine breite Palette von Anwendungen eingesetzt, von der Web- und Spieleentwicklung,
bis hin zum maschinellen Lernen und der wissenschaftlichen Datenverarbeitung.

\subsubsection{Vorteile von Python}
\paragraph{Einfache Syntax}
Die Syntax von Python ist so gestaltet, dass sie lesbar und unkompliziert ist. 
Die Sprache bietet klaren, logischen, Code für Projekte jeder Größe.

\paragraph{Umfangreiche Bibliotheksunterstützung}
Python ist bekannt für seine umfangreiche Bibliotheksunterstützung, die komplexe Aufgaben vereinfacht. 
Bibliotheken wie Pygame für die Spieleentwicklung und OpenCV für die Computer Vision haben die Entwicklungszeit für dieses Projekt erheblich reduziert.

\paragraph{Gemeinschaft und Ressourcen}
Mit einer großen und aktiven Community bietet Python viele von Ressourcen und Hilfestellungen. 
Die Open-Source-Natur von Python verbessert die Verfügbarkeit von Codebasen für die Programmierung. 
Dies, gepaart mit einer großen Anzahl von Tutorials und Dokumentationen, macht Python zu einer attraktiven Option für Entwickler.

\paragraph{Python und Schachprogrammierung}
Python hat ebenfalls einen Einsatz in der Schachprogrammierung gefunden. 
Engines wie Stockfish haben Implementierungen in Python. Die Flexibilität von Python und die einfache Erstellung von Prototypen 
machen es geeignet für die iterative Natur der Spieleentwicklung.

\subsubsection{Nachteile von Python}
\paragraph{Geschwindigkeit}
Die Ausführungsgeschwindigkeit von Python ist langsamer als bei kompilierten Sprachen wie C++ oder Java. 
Dies liegt daran, dass Python eine interpretierte Sprache ist, was bedeutet, dass sie Code Zeile für Zeile ausführt, 
was langsamer ist als andere Sprachen, die das gesamte Programm vor der Ausführung kompilieren.

\paragraph{Speicherverbrauch}
Auch der Speicherverbrauch von Python ist höher als bei anderen Sprachen.
Python verwendet dynamische Typisierung. Das heißt, dass Variablentypen nicht explizit vom Programmierer deklariert werden. 
Stattdessen wird der Typ zur Laufzeit bestimmt. In einer dynamisch typisierten Sprache kann jeder Variable jeder Typ zugewiesen werden. 
Zum Beispiel kann einer Variable, der anfangs ein String-Wert zugewiesen wird, später ein Integer oder ein Boolean zugewiesen werden.
Deshalb verwendet Python mehr Speicher, was zu Problemen führen kann.

\subsubsection{Verwendung in der Industrie}
Obwohl Python für Entwicklung und Prototyping beliebt ist, wird es in der Spieleindustrie nicht so weit verbreitet eingesetzt wie C++. 
Dies ist zum Teil auf die langsamere Geschwindigkeit und den höheren Speicherverbrauch zurückzuführen.

Zusammenfassend lässt sich sagen, dass Python für dieses Projekt aufgrund seiner Lesbarkeit, 
umfangreichen Bibliotheksunterstützung und der aktiven Community gewählt wurde. 
Trotz seiner Nachteile überwogen die Vorteile für die spezifischen Anforderungen dieser Anwendung.

\subsection{Alternativen zu Python}
Im Bereich der Programmierung gibt es eine Vielzahl von Sprachen, die genutzt werden können und jeweils 
einzigartige Stärken und Schwächen aufweisen. Obwohl sich für Python entschieden wurde, standen auch Java und C++
zur Debatte.

\subsubsection{Java}
Java ist seit langem ein fester Bestandteil in der Softwarebranche aufgrund ihrer plattformunabhängigen Natur, die durch die \ac{JVM} ermöglicht wird. 
Das bedeutet, dass ein Java-Programm auf jedem Gerät laufen kann, das eine \ac{JVM} hat, was zum Erfolg von Java über Plattformen hinweg beiträgt.

Die Vorteile von Java sind Robustheit, Plattformunabhängigkeit und eine umfangreiche Standardbibliothek, ähnlich wie bei Python. 
Es ist statisch typisiert, was mehr Sicherheit zur Kompilierzeit bietet. Darüber hinaus ist Java schneller als Python, da der Java-Code in 
Bytecode kompiliert wird und auf der \ac{JVM} läuft, ein Merkmal, das Java einen Vorteil bei Anwendungen bietet, die eine hohe Leistung erfordern.

Im Kontext der Schachprogrammierung kann die hohe Ausführungsgeschwindigkeit von Java zu effizienteren Suchalgorithmen und KI-Logik führen. 
Die Java-Syntax kann jedoch den Programmierprozess mühsamer machen und die Entwicklung von GUIs kann im Vergleich zu Python komplexer sein.

\subsubsection{C++}
C++ ist bekannt für ihre hohe Leistung und die Kontrolle über Systemressourcen. 
Sie wird häufig in Anwendungen eingesetzt, bei denen die Leistung kritisch ist, wie zum Beispiel in der Spieleentwicklung, 
in Betriebssystemen und in Echtzeitsystemen.

In der Schachprogrammierung kann C++ eine überlegene Leistung bieten und ermöglicht tiefere Suchen im Spielbaum innerhalb der gleichen Zeitspanne. 
Es ermöglicht auch Manipulationen auf niedrigerer Ebene, was beim Optimieren von Speicher und Rechenressourcen von Vorteil sein kann.
Die Speicherverwaltung in C++ ist manuell und kann zu Fehlern führen, die schwer zu entdecken sind.

\subsection{Vergleich von Python, Java und C++}
Obwohl alle drei Sprachen erfolgreich in der Schachprogrammierung eingesetzt wurden, können ihre Unterschiede in Syntax, 
Leistung und Nutzungsphilosophie zu unterschiedlichen Erfahrungen im Entwicklungsprozess führen. 
Python, mit seinem Fokus auf Code-Lesbarkeit und Einfachheit, führt zu schnelleren Entwicklungszyklen, 
kann jedoch in Bezug auf die Leistung hinterherhinken. Java und C++ bieten bei höherer Komplexität eine bessere Leistung.

Die Wahl der richtigen Sprache für ein Projekt hängt von den spezifischen Bedürfnissen und Einschränkungen des Projekts, 
der Vertrautheit des Entwicklungsteams mit der Sprache und den Anforderungen an Leistung und Plattformkompatibilität ab.

\section{Diskussion über Bibliotheken/Frameworks}
\subsubsection{Pygame}
Pygame ist eine Open-Source Python-Bibliothek, die für die Erstellung von Videospielen konzipiert ist. 
Sie basiert auf der \ac{SDL} und läuft auf fast jeder Plattform und jedem Betriebssystem.

Einer der Hauptvorteile von Pygame ist seine Einfachheit.
Pygame ist sehr flexibel, was dem Entwickler eine vollständige Kontrolle über den Spielentwicklungsprozess ermöglicht. 
Darüber hinaus verfügt es über eine große und aktive Community, die eine Vielzahl von Ressourcen, Tutorials und Anleitungen bereitstellt, 
die den Entwicklungsprozess weiter unterstützen.

\subsubsection{OpenCV}
OpenCV ist eine Open-Source Bibliothek für Computer Vision und maschinelles Lernen. 
Sie enthält eine Vielzahl an Algorithmen für Bildanalyse, Videoanalyse und Objekterkennung.

Der Funktionsumfang macht es für fast jede Aufgabe der Computer Vision geeignet. 
Es ist auf Recheneffizienz ausgelegt, mit einem starken Fokus auf Echtzeitanwendungen. 
Da es sich um ein Open-Source-Projekt handelt, gibt es eine große Community von Entwicklern, 
die kontinuierlich an Verbesserungen und Updates arbeiten und so eine starke Unterstützung bieten.

Auf der anderen Seite steigt mit der Funktionalität auch die Komplexität. Obwohl es eine Vielzahl von Funktionen bietet, 
kann die Nutzung dieser Funktionen kompliziert sein und das tiefgreifende Verständnis der Bibliothek erfordert einen erheblichen Aufwand und Zeit.

\subsubsection{Alternativen zu Pygame und OpenCV}
Neben Pygame und OpenCV, gibt es zahlreiche andere Bibliotheken und Frameworks, die für die Spielentwicklung und 
Bildverarbeitungsaufgaben verwendet werden könnten. Beispielsweise sind Unity und Unreal Engine leistungsstarke Engines, 
aber sie sind komplexer und erfordern eine erhebliche Lerninvestition. 
Im Bereich der Bildverarbeitung und Computer Vision könnten auch Bibliotheken wie die Python Imaging Library oder SciKit-Image verwendet werden, 
obwohl sie möglicherweise nicht die gleiche Breite an Funktionen wie OpenCV bieten.
Die Wahl der Bibliothek oder des Frameworks hängt von verschiedenen Faktoren ab, einschließlich der spezifischen Anforderungen des Projekts, 
der Vertrautheit der Entwickler mit den Tools und dem gewünschten Gleichgewicht zwischen Benutzerfreundlichkeit und Funktionalität.

Aufgrund der einfachen Implementierung wurde sich in diesem Projekt für Pygame entschieden. OpenCV wurde aufgrund von Vorwissen und Vertrautheit gewählt.

\section{Benutzeroberfläche und Benutzererfahrung}
Das \ac{UI} und die \ac{UX} sind entscheidende Aspekte jeder Anwendung. Eine ansprechende, intuitive und benutzerfreundliche Umgebung 
kann den Unterschied ausmachen zwischen einer Anwendung, die schnell verworfen wird, und einer, die regelmäßig genutzt und weiterempfohlen wird.

\subsubsection{Bedeutung einer benutzerfreundlichen Umgebung}
Bei der Erstellung einer Schachanwendung ist die Zugänglichkeit ein wesentlicher Aspekt. Eine benutzerfreundliche Oberfläche, 
die leicht navigierbar ist und deutlich beschriftete Optionen und Werkzeuge aufweist, ermöglicht es Spielern, 
effizient und effektiv mit dem Spiel zu interagieren. Darüber hinaus spielt die Zugänglichkeit eine wichtige Rolle bei der Gewährleistung, 
dass die Anwendung für alle potenziellen Nutzer zugänglich ist.

Die Nutzerzufriedenheit ist ein entscheidender Faktor, der untrennbar mit einer benutzerfreundlichen Umgebung verbunden ist. 
Wenn Nutzer die Anwendung als schwer zu bedienen oder zu navigieren empfinden oder wenn sie ihren Erwartungen hinsichtlich 
Funktionalität und Ästhetik nicht entspricht, führt das unweigerlich zu Unzufriedenheit. 
Diese Unzufriedenheit kann zu negativen Bewertungen und Feedback führen, was die Reputation und 
Nutzerbasis der Anwendung beeinträchtigen kann.

\subsubsection{Designprinzipien für eine benutzerfreundliche Umgebung}
\paragraph{Einfachheit}
Eine Benutzeroberfläche sollte so einfach wie möglich sein und den Benutzern nur die notwendigen Funktionen zur Verfügung stellen, 
um ihre Aufgaben zu erfüllen. Dies bedeutet nicht, die Anwendung von nützlichen Funktionen zu befreien, sondern diese so zu präsentieren, 
dass sie den Benutzer nicht überfordern.

\paragraph{Konsistenz}
Konsistenz in Designelementen wie Schriftarten, Farben und Layout reduziert die Lernkurve für Benutzer. 
Es ermöglicht ihnen, sich schnell mit der Benutzeroberfläche vertraut zu machen, was zu einer und intuitiven Benutzererfahrung führt.
 
\paragraph{Rückmeldung und Reaktionszeit}
Benutzern sollte eindeutiges Feedback zu ihren Aktionen gegeben werden. 
Wenn beispielsweise ein Benutzer einen Zug in einem Schachspiel macht, sollte die Anwendung schnell reagieren und die Ergebnisse dieses Zuges anzeigen. 
Verzögertes oder unzureichendes Feedback kann zu Benutzerfrustration und Verwirrung führen.~\cite{Uxpin_2022_uxpin}

\paragraph{Benutzerkontrolle}
Benutzer sollten das Gefühl haben, die Anwendung zu kontrollieren. Dies kann erreicht werden, indem Benutzern erlaubt wird, 
Aktionen einfach rückgängig zu machen und zu wiederholen, sich frei innerhalb der Anwendung zu bewegen und 
Einstellungen nach ihren Vorlieben anzupassen.~\cite{Uxpin_2022_uxpin}

\section{Schlussfolgerung}
Aufgrund der einfachen Erstellung von Prototypen, die eine iterative Arbeitsweise fördern, 
wurde sich in diesem Projekt für Python als Programmiersprache entschlossen. Die schwächere Leistung ist zwar bemerkbar, jedoch immernoch 
gut genug um die Ziele dieses Projektes zu erreichen.

Aufgrund der Implementierung von Python wurde sich für die Nutzung von Pygame als Game-Engine entschieden, um schnell und unkompliziert 
Prototypen zu erstellen, die der Grund für die Wahl der Programmiersprache waren.

Die Verbindung zur Kamerafunktionalität ist im Projekt mittels OpenCV verankert, da hier, im Gegensatz zu anderen Moeglichkeiten, schon Vorkenntnisse 
herrschen.

